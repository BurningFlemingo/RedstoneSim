% template provided by https://joshldavis.com/2014/02/12/doing-your-homework-in-latex/


\documentclass{article}

\usepackage{fancyhdr}
\usepackage{extramarks}
\usepackage{amsmath}
\usepackage{amssymb}
\usepackage{amsthm}
\usepackage{amsfonts}
\usepackage{tikz}
\usepackage{enumitem}
\usepackage{physics}
\usepackage{tasks}
\usepackage{mathtools}
\usepackage{outlines}
\usepackage{algorithm}
\usepackage{algpseudocode}
\usepackage[most]{tcolorbox}
\usepackage{multicol}

\newtheorem{problem}{Problem}[section]
\newtheorem{defn}{Definition}[section]
\newtheorem{theorem}{Theorem}[section]

\usetikzlibrary{automata,positioning,graphs,graphs.standard,quotes}
 
\topmargin=-0.45in
\evensidemargin=0in
\oddsidemargin=0in
\textwidth=6.5in
\textheight=9.0in
\headsep=0.25in

\linespread{1.1}

\pagestyle{fancy}
\lhead{\hmwkAuthorName}
\chead{\hmwkClass: \hmwkTitle}
\rhead{\hmwkDueDate}
\lfoot{\lastxmark}
\cfoot{\thepage}

\renewcommand\headrulewidth{0.4pt}
\renewcommand\footrulewidth{0.4pt}

\setlength\parindent{0pt}

%
% Create Problem Sections
%

%
% Homework Details
%   - Title
%   - Due date
%   - Class
%   - Section/Time
%   - Instructor
%   - Author
%

\newcommand{\hmwkTitle}{Minecraft Redstone with Graph and Set Theory}
\newcommand{\hmwkDueDate}{June 26, 2024}
\newcommand{\hmwkClass}{ABM}
\newcommand{\hmwkClassInstructor}{Mr. Rios}
\newcommand{\hmwkAuthorName}{\textbf{John Fleming}}

%
% Title Page
%

\title{
    \vspace{2in}
    \textmd{\textbf{\hmwkClass:\ \hmwkTitle}}\\
    \normalsize\vspace{0.1in}\small{Due on \hmwkDueDate}\\
    \vspace{0.1in}\large{\textit{\hmwkClassInstructor\ }}
    \vspace{3in}
}

\author{\hmwkAuthorName}
\date{}

\renewcommand{\part}[1]{\textbf{\large Part \Alph{partCounter}}\stepcounter{partCounter}\\}

%
% Various Helper Commands
%

% For derivatives
\newcommand{\deriv}[1]{\frac{\mathrm{d}}{\mathrm{d}x} (#1)}

% For partial derivatives
\newcommand{\pderiv}[2]{\frac{\partial}{\partial #1} (#2)}

% Alias for the Solution section header
\newcommand{\solution}{\textbf{\large Solution}}

\newtcolorbox{minbox}[1][]{
	boxrule=0.75pt,
	boxsep=0pt,
	sharp corners,  % Square edges
	colframe=black,  % Set the color of the outline
	colback=white,  % Set the color of the fill
	coltext=black,
	lefttitle=0pt,
	colbacktitle=white,
	coltitle=black,
	after={\bigskip},
	#1
}


% \begin{enumerate}[label=\(\textbf{Problem } \mathbf{\arabic*.}\), align=left]
% \end{enumerate}

% \begin{enumerate}[label=\(\mathbf{\alph*)}\), align=left]
% \end{enumerate}

% \begin{displaymath}
% 	\begin{array}{|c c|c|}
% 	P & Q & \neg P \vee Q\\
% 	\hline
% 	F & F & T\\
% 	F & T & T\\
% 	T & F & F\\
% 	T & T & T\\
% 	\end{array}
% \end{displaymath}

% \begin{minbox}[ams align*]
% \end{minbox}

% \begin{center}
% 	\fbox{
% 	}
% \end{center}

\begin{document}
\maketitle

\pagebreak

\section{Notation}
\noindent

\begin{multicols}{2}
\subsection*{Standard}
\begin{description}
    \item[\(\implies\)] implies
    \item[\(\iff\)] if and only if 
    \item[\(\varnothing\)] empty set
    \item[\(\cup\)] union 
    \item[\(\cap\)] intersection
    \item[\(A \subseteq B\)] improper subset
    \item[\(A \to B\)] maps to
    \item[\(A \setminus B\)] difference of sets 
    \item[\(A \coloneq B\)] is defined to be 
    \item[\(A \times B\)] Cartesian product (see Definition \ref{defn:cartesian_product})
	\item[\((a, b)\)] ordered pair (see Definition \ref{defn:pair})
    \item[\((n_{1}, n_{2}, \ldots, n_{n})\)] tuple (see Definition \ref{defn:pair})
    \item[\(\mathcal{P}(A)\)] power set
	\item[\(f : A \to B\)] function mapping (see Definition \ref{defn:function})
    \item[\(B^A\)] set of functions (see Definition \ref{defn:function_sets})
	\item[\(\pi_{1}, \pi_{2}\)] coordinate projection functions (see Definition \ref{defn:coord_projection})
    \item[\(F|_{C}\)] function restriction (see Definition \ref{defn:function_restriction})
    \item[\(1_A\)] indicator function (see Definition \ref{defn:indicator_function})
	\item[\(\mathbb{N}_{0}\)] set of natural numbers including zero
	\item[\(\mathbb{Z}^+\)] set of positive integers not including zero(see Definition \ref{defn:number_sets})
	\item[\(\mathbb{Z}^-\)] set of negative integers not including zero (see Definition \ref{defn:number_sets})
	\item[\(\mathbb{Z}_{0}^-\)] set of negative integers including zero (see Definition \ref{defn:number_sets})
\end{description}

\columnbreak

\subsection*{Non-Standard}
\begin{description}
	\item[\(max\)] function mapping sets to their largest element (see Definition \ref{defn:max})
    \item[\(S\)] set of possible signal strengths (see Definition \ref{defn:signals})
    \item[\(S^+\)] set of positive signal strengths (see Definition \ref{defn:signals})
    \item[\(S_{0}\)] set of zero signal strength (see Definition \ref{defn:signals})
    \item[\(C\)] set of supported redstone components (see Definition \ref{defn:components})
    \item[\(\Phi_{G_{r}}\)] function mapping \(C\) to every state update function (see Definition \ref{defn:behavior_functions})
    \item[\(\Omega\)] function mapping \(C\) to every output function (see Definition \ref{defn:behavior_functions})
    \item[\(\Psi_{G}\)] propagation function (see Definition \ref{defn:propegation})
    \item[\(I_{G_{r}}\)] input function (see Definition \ref{defn:input_function})
\end{description}
\end{multicols}

\section{Introduction}
In this model, we attempt to formalize Minecraft Redstone using only Zermelo-Fraenkel set theory with the axiom of choice. 

\section{Foundations}
\noindent

\begin{defn} (Number Set Notation)
	\label{defn:number_sets}
	\begin{align}
		\mathbb{Z}^+ &\coloneq \mathbb{N}_{0} \setminus \{0\} \\
		\mathbb{Z}^- &\coloneq \{n \in \mathbb{Z}\mid n < 0\} \\
		\mathbb{Z}_{0}^- &\coloneq \mathbb{Z}^- \cup \{0\}
	\end{align}
\end{defn}

\medskip

\begin{defn} (Kuratowski Pair)
	\label{defn:pair}
	
	Let \(a\) and \(b\) be any elements. \\
	Then the ordered pair between them
	\begin{equation}
		(a, b) \coloneq \{\{a\}, \{a, b\}\}
	\end{equation}
	where for any elements \(a_{1}, a_{2}, \ldots, a_{n}\),
	\begin{equation}
		(a_{1}, (a_{2}, \ldots, (a_{n - 1}, a_{n}))) \coloneq (a_{1}, a_{2}, \ldots, a_{n})
	\end{equation}
	and 
	\begin{equation}
		((a_{1}, \ldots, (a_{n - 2}, a_{n - 1})), a_{n}) \coloneq (a_{1}, a_{2}, \ldots, a_{n})
	\end{equation}

\end{defn}

\medskip

\begin{defn} (Cartesian Product)
	\label{defn:cartesian_product}

	Let \(A\) and \(B\) be any sets. \\
	Then
	\begin{equation}
		A \times B \coloneq \{(a, b)) \mid \exists a \in A, \exists b \in B\}
	\end{equation}
	where for any sets \(A_{1}, A_{2}, \ldots A_{n}\),
	\begin{equation}
		A_{1} \times A_{2} \times \cdots \times A_{n} \coloneq\{(a_{1}, a_{2}, \ldots, a_{n}) \mid \exists a_{1} \in A_{1}, 
			\exists a_{2} \in A_{2}, \ldots \exists a_{n} \in A_{n}\} 
	\end{equation}
	
\end{defn}

\medskip

\begin{defn} (Functions)
	\label{defn:function}
	
	Let \(A\) and \(B\) be any sets. \\
	Then a function \(f : A \to B\) if
	\begin{equation}
		f \subseteq A \times B
	\end{equation}
	\begin{equation}
		\forall a \in A, \exists b \in B((a, b) \in f)
	\end{equation}
	and
	\begin{equation}
		\forall a \in A \forall b, b' \in B((a, b) \in f \land (a, b') \in f \implies b = b')
	\end{equation}
	Now let a function \(f : A \to B\), then \(f(x)\) for any \(x \in A\) is defined such that
	\begin{equation}
		f(x) = y \iff (x, y) \in f
	\end{equation}

	Next, let \(A\) and \(B\) be any sets where \(A = A_{1} \times A_{2} \times \cdots \times A_{n}\), and let 
	a function \(f : A \to B\). \\
	Then \(f(x_{1}, x_{2}, \ldots, x_{n})\) where \((x_{1}, x_{2}, \ldots, x_{n}) \in A\) is defined such that 
	\begin{equation}
		f(x_{1}, x_{2}, \ldots, x_{n}) = f(x), x \in A
	\end{equation}
\end{defn}

\medskip

\begin{defn} (Coordinate Projection)
	\label{defn:coord_projection}
	
	Let \(A\) and \(B\) any sets. \\
	Then the function
	\begin{equation}
		\pi_{1} : A \times B \to A
	\end{equation}
	where
	\begin{equation}
		\pi_{1}(a, b) = \bigcup \bigcap (a, b)
	\end{equation}

	Furthermore, given \(A\) and \(B\) are any sets, then the function
	\begin{equation}
		\pi_{2} : A \times B \to B
	\end{equation}
	where
	\begin{equation}
		\pi_{2}(a, b) = \bigcup\{a \in \bigcup (a, b)\mid a \notin \bigcap(a, b)\}
	\end{equation}
\end{defn}

\medskip

\begin{defn} (Set of Functions)
	\label{defn:function_sets}
	
	Let \(A\) and \(B\) be any sets. \\
	Then 
	\begin{equation}
		B^A \coloneq \{f \in \mathcal{P}(A \times B) \mid f : A \times B \}
	\end{equation}
\end{defn}

\begin{defn} (Function Restriction)
	\label{defn:function_restriction}
	
	Let \(A\), \(B\), and \(C\) any sets where \(C \subseteq A\), and let a function \(f : A \to B\). \\
	Then the function
	\begin{equation}
		f|_{C} : C \to B
	\end{equation}
	where 
	\begin{equation}
		f|_{C}(x) = f(x), x \in C
	\end{equation}
\end{defn}

\medskip

\begin{defn} (Indicator Function)
	\label{defn:indicator_function}
	
	Let \(X\) and \(A\) be any \(2\) sets where \(A \subseteq X\). \\ 
	Then the function
	\begin{equation}
		1_{A} : X \to \{0, 1\}
	\end{equation}
	where
	\begin{equation}
		1_{A}(x) = \begin{cases}
			1 & \text{if } x \in A \\
			0 & \text{if } x \notin A
		\end{cases}
	\end{equation}
\end{defn}

\medskip

\begin{defn} (Max)
	\label{defn:max}
	
	Let \(Y\) be any non-empty totally ordered set. \\
	Then the function
	\begin{equation}
		max : \mathcal{P}(Y) \to Y
	\end{equation}
	where
	\begin{equation}
		\forall X \in \mathcal{P}(Y), \exists x \in X(max(X) = x \iff \forall y \in X(x \geq y))
	\end{equation}
\end{defn}


\medskip

\begin{defn} (Vertices)
	
	Let \(V\) be a set. \\
	Then \(V\) is a set of vertices if
	\begin{equation}
		V \neq \varnothing
	\end{equation}
\end{defn}

\medskip

\begin{defn} (Directed Edges)
	
	Let \(V\) be any set of vertices. \\
	Then \(E\) is a set of directed edges if 
	\begin{equation}
		E \subseteq (V \times V)
	\end{equation}
\end{defn}

\medskip

\begin{defn} (Graph)
	
		Let \(V\) be any set of vertices, and let \(E\) be any set of directed edges on \(V\). \\
		Then \(G\) is a digraph if
		\begin{equation}
			G = (V, E)
		\end{equation}
\end{defn}

\section{Redstone Model}

\begin{defn} (Signal Sets)
	\label{defn:signals}
	
	\begin{align}
		S &\coloneq \{0, 1, 2, ..., 15\} \\
		S^+ &\coloneq \{x \in S \mid x > 0\} \\
		S_{0} &\coloneq \{0\}
	\end{align}
\end{defn}

\medskip

\begin{defn} (Propagation)
	\label{defn:propegation}
	
Let \(G = (V, E)\) be any digraph. \\
Then the function
\begin{equation}
	\Psi_{G} : V \rightarrow \mathcal{P}(V) 
\end{equation}
where
\begin{equation*}
	\Psi_{G}(v) = 
	\{v \in V \mid \exists (u, v) \in E\}
\end{equation*}
\end{defn}

\medskip

\begin{defn} (Components)
	\label{defn:components}
	\begin{equation}
		C \coloneq \{R_{1_c}, R_{2_c}, R_{3_c}, R_{4_c}, T_{c}\}
	\end{equation}
	where 
	\begin{outline}
		\1 \(R_{n_{c}}\) for any \(n \in {1, 2, 3, 4}\) corresponds to a repeater with a delay of \(n\) ticks.
		\1 \(T_{c}\) corresponds to a redstone torch.
	\end{outline}
\end{defn}

\pagebreak

\begin{defn} (Redstone Digraph)
	
	Let
	
	\begin{outline}
		\1 \(G = (V, E)\) be any digraph
		\1 \(\Sigma : V \times \mathbb{N}_{0} \to \mathbb{N}_{0}\)
		\1 \(\lambda : V \to C\)
		\1 \(\mu : E \to S \times S\)
	\end{outline}

	where 
	\begin{outline}
		\1 \(\Sigma\) maps vertices and ticks to a numeric state at that tick.
		\1 \(\lambda\) maps vertices to components.
		\1 \(\mu\) maps edges to the signal droppoff between the tail and head vertices where the output of the
		tail vertex is the input to the state of the head vertex.
			\2 note: a droppoff of \(15\) implies there is no signal going into the head vertex (i.e. its disconnected, 
			and therefore floating, which defaults to zero in Minecraft), while a dropoff of 
			\(0\) implies the vertices as components are touching,
	\end{outline}

	Then \(G_{r}\) is a redstone digraph if 
	\begin{equation}
	G_{r} = (G, \Sigma, \lambda, \mu)
	\end{equation}
\end{defn}

\medskip

\begin{defn} (Behavior Functions)
	\label{defn:behavior_functions}
	
	Let \(G_{r} = (G, \Sigma, \lambda, \mu)\) be any redstone digraph where \(G = (V, E)\).
	Then the function
	\begin{equation}
		\Phi_{G_{r}} : C \to \mathbb{N}_{0}^{V \times S \times \mathbb{N}_{0} \times \mathbb{Z}^+}
	\end{equation}
	and the function
	\begin{equation}
		\Omega : C \to  S^{\mathbb{N}_{0}}
	\end{equation}

	Where
	\begin{outline}
		\1 \(\Phi_{G_{r}}\) maps components to state "update" functions.
			\2 note: the update functions take in
				\3 the vertex of the component.
				\3 the back input.
				\3 the numeric internal state of the component at some tick \(t\).
				\3 some tick \(t\),
				\4 note: the tick \(t\) is a positive integer because the state at \(t = 0\) should be manually defined.
			\2 note: the update functions map the input to the internal state of that vertex 
			at the next tick.
		\1 \(\Omega\) maps components to "output" functions.
			\2 note: the output function takes in
				\3 the numeric internal state of the component 
			\2 note: the output function maps the input to what output would be based on its state, i.e. it calculates 
			the output for the current tick.
	\end{outline}
\end{defn}

\medskip

\begin{defn} (Input)
	\label{defn:input_function}
	
	Let \(G_{r} = (G, \Sigma, \lambda, \mu)\) be any redstone digraph where \(G = (V, E)\).
	Then the function
	\begin{equation}
		I_{G_{r}} : V \times \mathbb{Z}^+ \times \{1, 2\} \to S
	\end{equation}
	where
	\begin{equation}
		I_{G_{r}}(v, t, i) = 
		\begin{cases}
			0 & \text{if } \Psi_{G}(v) = \varnothing \\
			max(\{ \Omega(\lambda(u))(\Sigma(u, t)) - \pi_{1}(\mu(u, v)) \in S \mid u \in \Psi_{G}(v)  \} \cup \{0\}) & \text{if } i = 0 \land \Psi_{G}(v) \neq \varnothing \\
			max(\{ \Omega(\lambda(u))(\Sigma(u, t)) - \pi_{2}(\mu(u, v)) \in S \mid u \in \Psi_{G}(v)  \} \cup \{0\}) & \text{otherwise}
		\end{cases}
	\end{equation}

	Note	
	\begin{outline}
		\1 \(v\) is the vertex to get the input of.
		\1 \(t\) is time in ticks to get the input at. 
		\1 \(i\) is used to multiplex between the back input and side input of a vertex.
	\end{outline}
\end{defn}
	
\begin{defn} (State Update)

Let \(G_{r} = (G, \Sigma, \lambda, \mu)\) be any redstone digraph where \(G = (V, E)\).

Then, for any vertex \(v \in V\) at an arbitrary tick \(t \in \mathbb{N}_{0}\),

\begin{equation}
	\Sigma(v, t + 1)|_{V \times \mathbb{Z}^+} = \Phi_{G_{r}}(\lambda(v))(v, I_{G_{r}}(v, t, 1), \Sigma(v, t), t) \\
\end{equation}

Note: the state mapping isn't defined for \(t = 0\) to allow the implementation to define a custom mapping for all 
vertices at \(t = 0\).

\end{defn}

\section{Redstone Component Output Functions}

\begin{defn} (Redstone Torch)

	Let \(G_{r} = (G, \Sigma, \lambda, \mu)\) be any redstone digraph where \(G = (V, E)\).
	Next let
	\begin{equation}
		\phi(v, i, \sigma, t) = i 
	\end{equation}
	and
	\begin{equation}
		\omega(\sigma) = 15 * 1_{S_{0}}(\sigma)
	\end{equation}
	Then \(\Phi_{G_{r}}(T_{c}) = \phi\) and \(\Omega(T_{c}) = \omega\).
\end{defn}

\begin{defn} (Repeater(s))
	
	Let
	\begin{outline}
	\1 \(G_{r} = (G, \Sigma, \lambda, \mu)\) be any redstone digraph where \(G = (V, E)\).
	\1 \(t \in \mathbb{N}_{0}\).
	\1 \(i_{i} = max(\{I_{G_{r}}(u, t, 2) \mid u \in \Psi_{G}\} \cup \{0\})\)
	\end{outline}
	Next, for \(n \in \{1, 2, 3, 4\}\), let
	\begin{align}
		& N_{n}(v, i, \sigma, t) = 
		\begin{cases}
			1		   & \text{if }	 (i \in S^+) \land [(\sigma \geq 2n) \vee (\sigma = 0)]\\
			\sigma + 1 & \text{if }	 0 < \sigma < n \\
			\sigma + 1 & \text{if }	 (i \in S_{0}) \land (n \leq \sigma < 2n) \\
			n		   & \text{if }	 (i \in S^+) \land (\sigma = 2n - 1) \\
			0		   & \text{otherwise}
		\end{cases}
	\end{align}
	\begin{align}
		& L_{n}(v, i, \sigma, t) = 
		\begin{cases}
			n & \text{if }	n <= \sigma < 2n \\
			0 & \text{otherwise} \\
		\end{cases}
	\end{align}
	\begin{align}
		& \phi_{n}(v, i, \sigma, t) = 
		\begin{cases}
			L_{n}(v, i, \sigma, t) & \text{if } (i_{i} > 0) \vee (I_{G_{r}}(v, t, 2) > 0)\\
			N_{n}(v, i, \sigma, t) & \text{otherwise}
		\end{cases}
	\end{align}	
	and
	\begin{align}
		& \omega_{n}(\sigma) = 
		\begin{cases}
			15 & \text{if } n \leq \sigma < 2n \\
			0 & \text{ otherwise }
			\end{cases}
	\end{align}
	Then, \(\Phi_{G_{r}}(R_{n_c}) = \phi_{n}\) and \(\Omega(R_{n_{c}}) = \omega_{n}\).
\end{defn}

\section{Applications}

\begin{proof} Clock proof
	
	\begin{outline}
		\1 Let \(G_{r} = (G, \Sigma, \lambda)\) be a redstone digraph where \(G = (V, E)\). 
		\1 Let \(V = \{t_{1}, t_{2}, t_{3}\}\) and \(E = \{(t_{1}, t_{2}), (t_{2}, t_{3}), (t_{3}, t_{1})\}\).
		\1 Let \(\forall v \in V(\Sigma(v, 0) = 0)\).
		\1 Let \(\forall v \in V(\lambda(v) = T_{c})\).
		\1 Let \(\mu(v_{1}, v_{2}) = (2, 15)\), \(\mu(v_{2}, v_{3}) = (2, 15)\), \(\mu(v_{3}, v_{1}) = (2, 15)\).
	\end{outline}

	Assuming that this redstone circuit was not built instantly
	Let \(v = t_{3}\) and \(t = 1\). \\
	Then
	\begin{align*}
		\Sigma(v, t + 1) &= \Phi_{G_{r}}(\lambda(v))(v, I_{G_{r}}(v, t, 1), \Sigma(v, t), t) \implies \\
		\Sigma(v, t) &= \Phi_{G_{r}}(\lambda(v))(v, I_{G_{r}}(v, t - 1, 1), \Sigma(v, t - 1), t - 1) \implies \\
		\Sigma(t_{3}, 1) &= \Phi_{G_{r}}(\lambda(v_{1}))(v_{1}, I_{G_{r}}(v_{1}, 1 - 1, 1), \Sigma(v_{1}, 1 - 1), 1 - 1)\\
						 &= \Phi_{G_{r}}(\lambda(v_{1}))(v_{1}, I_{G_{r}}(v_{1}, 0, 1), \Sigma(v_{1}, 0), 0) \\
						 &= \Phi_{G_{r}}(T_{c})(v_{1}, I_{G_{r}}(v_{1}, 0, 1), 0, 0) \\
						 &= \Phi_{G_{r}}(T_{c})(v_{1}, max(\{\Omega(\lambda(v_{2}))(\Sigma(v_{2}, 0)) - \pi_{1}(\mu(v_{2}, v_{3}))\} \cup \{0\}), 0, 0) \\
						 &= \Phi_{G_{r}}(T_{c})(v_{1}, max(\{\Omega(T_{c})(0) - \pi_{1}(2, 15)\} \cup \{0\}), 0, 0) \\
						 &= \Phi_{G_{r}}(T_{c})(v_{1}, max(\{15 - 2\} \cup \{0\}), 0, 0) \\
						 &= \Phi_{G_{r}}(T_{c})(v_{1}, 13, 0, 0) \\
						 &= 15 \\
	\end{align*}

	Next let \(v = t_{3}\) and \(t = 2\). \\
	Then 
	\begin{align*}
		\Sigma(t_{3}, 2) &= \Phi_{G_{r}}(\lambda(v_{1}))(v_{1}, I_{G_{r}}(v_{1}, 2 - 1, 1), \Sigma(v_{1}, 2 - 1), 2 - 1)\\
						 &= \Phi_{G_{r}}(\lambda(v_{1}))(v_{1}, I_{G_{r}}(v_{1}, 1, 1), \Sigma(v_{1}, 1), 1) \\
	\end{align*}
	where 
	\begin{align*}
		\Sigma(v_{1}, 1) &= \Phi_{G_{r}}(\lambda(v_{1}))(v_{1}, I_{G_{r}}(v_{1}, 1 - 1, 1), \Sigma(v_{1}, 1 - 1), 1 - 1) \\ 
						 &= \Phi_{G_{r}}(T_{c})(v_{1}, I_{G_{r}}(v_{1}, 0, 1), \Sigma(v_{1}, 0), 0) \\
						 &= \Phi_{G_{r}}(T_{c})(v_{1}, max(\{\Omega(\lambda(v_{3}))(\Sigma(v_{3}, 0)) - \pi_{1}(\mu(v_{3}, v_{1}))\} \cup \{0\}), 0) \\
						 &= \Phi_{G_{r}}(T_{c})(v_{1}, max(\{\Omega(T_{c})(0) - \pi_{1}(2, 15)\} \cup \{0\}), 0) \\
						 &= \Phi_{G_{r}}(T_{c})(v_{1}, max(\{15 - 2\} \cup \{0\}), 0, 1), 0\\
						 &= \Phi_{G_{r}}(T_{c})(v_{1}, 13, 0, 0) \\
						 &= 15 \\
	\end{align*}
	then 
	\begin{align*}
		\Phi_{G_{r}}(\lambda(v_{1}))(v_{1}, I_{G_{r}}(v_{1}, 1, 1), 13, 1) &= \Phi_{G_{r}}(T_{c})(v_{1}, I_{G_{r}}(v_{1}, 1, 1), 13, 1)\\
																		   &= \Phi_{G_{r}}(T_{c})(v_{1}, max(\{\Omega(\lambda(v_{3}))(\Sigma(v_{1}, 0)) - \pi_{1}(\mu(v_{1}, v_{3}))\} \cup \{0\}), 13, 1) \\
																		   &= \Phi_{G_{r}}(T_{c})(v_{1}, max(\{\Omega(T_{c})(13) - \pi_{1}(2, 15)\} \cup \{0\}), 13, 1)\\
																		   &= \Phi_{G_{r}}(T_{c})(v_{1}, max(\{0 - 2\} \cup \{0\}), 13, 1)\\
																		   &= \Phi_{G_{r}}(T_{c})(v_{1}, 0, 15, 1)\\
																		   &= \Phi_{G_{r}}(T_{c})(v_{1}, 0, 15, 1)\\
																		   &= 0\\
	\end{align*}

	The output of \(v_{3}\) at \(t = 2\) is then 
	\begin{align*}
		\Omega(\lambda(t_{3}))(\Sigma(t_{3}, 2)) &= \Omega(T_{c})(0) \\
		& = 15 \\ 
	\end{align*}
	then for \(t = 1\)
	\begin{align*}
		\Omega(\lambda(t_{3}))(\Sigma(t_{3}, 1)) &= \Omega(T_{c})(15) \\
		& = 0 \\ 
	\end{align*}

	and finally for \(t = 0\)
	\begin{align*}
		\Omega(\lambda(t_{3}))(\Sigma(t_{3}, 0)) &= \Omega(T_{c})(\Sigma(t_{3}, 0)) \\
		& = 15 \\ 
	\end{align*}

	Hence, it is then proven that

	\begin{displaymath}
		\begin{array}{|c|c|c|c|}
			t_{3} & t = 0 & t = 1 & t = 2\\
 		\hline
			0 & 15 & 0 & 15
 		\end{array}
	\end{displaymath}
\end{proof}

\end{document}

