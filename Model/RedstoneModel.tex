% template provided by https://joshldavis.com/2014/02/12/doing-your-homework-in-latex/


\documentclass{article}

\usepackage{fancyhdr}
\usepackage{extramarks}
\usepackage{amsmath}
\usepackage{amssymb}
\usepackage{amsthm}
\usepackage{amsfonts}
\usepackage{tikz}
\usepackage{enumitem}
\usepackage{physics}
\usepackage{tasks}
\usepackage{mathtools}
\usepackage[most]{tcolorbox}

\newtheorem{problem}{Problem}[section]
\newtheorem{defn}{Definition}[section]
\newtheorem{theorem}{Theorem}[section]

\usetikzlibrary{automata,positioning,graphs,graphs.standard,quotes}
 
\topmargin=-0.45in
\evensidemargin=0in
\oddsidemargin=0in
\textwidth=6.5in
\textheight=9.0in
\headsep=0.25in

\linespread{1.1}

\pagestyle{fancy}
\lhead{\hmwkAuthorName}
\chead{\hmwkClass: \hmwkTitle}
\rhead{\hmwkDueDate}
\lfoot{\lastxmark}
\cfoot{\thepage}

\renewcommand\headrulewidth{0.4pt}
\renewcommand\footrulewidth{0.4pt}

\setlength\parindent{0pt}

%
% Create Problem Sections
%

%
% Homework Details
%   - Title
%   - Due date
%   - Class
%   - Section/Time
%   - Instructor
%   - Author
%

\newcommand{\hmwkTitle}{Minecraft Redstone with Graph and Set Theory}
\newcommand{\hmwkDueDate}{June 26, 2024}
\newcommand{\hmwkClass}{ABM}
\newcommand{\hmwkClassInstructor}{Mr. Rios}
\newcommand{\hmwkAuthorName}{\textbf{John Fleming}}

%
% Title Page
%

\title{
    \vspace{2in}
    \textmd{\textbf{\hmwkClass:\ \hmwkTitle}}\\
    \normalsize\vspace{0.1in}\small{Due on \hmwkDueDate}\\
    \vspace{0.1in}\large{\textit{\hmwkClassInstructor\ }}
    \vspace{3in}
}

\author{\hmwkAuthorName}
\date{}

\renewcommand{\part}[1]{\textbf{\large Part \Alph{partCounter}}\stepcounter{partCounter}\\}

%
% Various Helper Commands
%

% For derivatives
\newcommand{\deriv}[1]{\frac{\mathrm{d}}{\mathrm{d}x} (#1)}

% For partial derivatives
\newcommand{\pderiv}[2]{\frac{\partial}{\partial #1} (#2)}

% Alias for the Solution section header
\newcommand{\solution}{\textbf{\large Solution}}

\newtcolorbox{minbox}[1][]{
	boxrule=0.75pt,
	boxsep=0pt,
	sharp corners,  % Square edges
	colframe=black,  % Set the color of the outline
	colback=white,  % Set the color of the fill
	coltext=black,
	lefttitle=0pt,
	colbacktitle=white,
	coltitle=black,
	after={\bigskip},
	#1
}


% \begin{enumerate}[label=\(\textbf{Problem } \mathbf{\arabic*.}\), align=left]
% \end{enumerate}

% \begin{enumerate}[label=\(\mathbf{\alph*)}\), align=left]
% \end{enumerate}

% \begin{displaymath}
% 	\begin{array}{|c c|c|}
% 	P & Q & \neg P \vee Q\\
% 	\hline
% 	F & F & T\\
% 	F & T & T\\
% 	T & F & F\\
% 	T & T & T\\
% 	\end{array}
% \end{displaymath}

% \begin{minbox}[ams align*]
% \end{minbox}

% \begin{center}
% 	\fbox{
% 	}
% \end{center}

\begin{document}
\maketitle

\pagebreak

\section{Foundations}

\begin{defn} (Max)
	
	Let \(X\) be any \(X \subseteq \mathbb{N}_{0}\). Then
	\begin{equation}
		Max(X) \coloneq \{x \in X \mid \forall y \in X(x \geq y)\}
	\end{equation}
\end{defn}

\begin{defn} (Number Set Notation)
	\begin{align}
		\mathbb{Z}^+ &\coloneq \mathbb{N}_{0} \setminus \{0\} \\
		\mathbb{Z}^- &\coloneq \{n \in \mathbb{Z}\mid n < 0\} \\
		\mathbb{Z}_{0}^- &\coloneq \mathbb{Z}^- \cup \{0\} \\
	\end{align}
\end{defn}

\medskip

\begin{defn} (Signal Sets)
	
	\begin{align}
		S &\coloneq \{0, 1, 2, ..., 15\} \\
		S^+ &\coloneq \{x \in S \mid x > 0\} \\
		S_{0} &\coloneq \{0\}
	\end{align}
\end{defn}

\medskip

\begin{defn} (Max Delay)
	
	\begin{equation}
		D_{m} \coloneq 4
	\end{equation}

	Max delay of any component.
\end{defn}

\medskip

\begin{defn} (Component Set)
	\begin{equation}
		C \coloneq \{T_{c}, R1_{c}, R2_{c}, R3_{c}, R4_{c}, O_{c}, I_{c}\}
	\end{equation}

	The set of supported redstone components. 
\end{defn}

\medskip

\begin{defn} (Vertices)
	
	Let \(V\) be a set. \\
	Then \(V\) is a set of vertices if
	\begin{equation}
		V \neq \varnothing
	\end{equation}
\end{defn}

\medskip

\begin{defn} (Directed Edges)
	
	Let \(V\) be any set of vertices. \\
	Then \(E\) is a set of directed edges on \(V\) if 
	\begin{equation}
		E \subseteq (V \times V)
	\end{equation}
\end{defn}

\medskip

\begin{defn} (Graph)
	
		Let \(V\) be any set of vertices, and let \(E\) be any set of directed edges on \(V\). \\
		Then \(G\) is a digraph on \(V\) and \(E\) if
		\begin{equation}
			G = (V, E)
		\end{equation}
\end{defn}

\medskip

\begin{defn} (Propegation)
	
Let \(G = (V, E)\) be any digraph. Then

\begin{equation}
	\omega^+ : V \times \mathcal{P}(G) \rightarrow \mathcal{P}(V) 
\end{equation}
Where
\begin{equation*}
	\omega^+(v, G) = 
	\{u \in V \mid (u, v) \in E\}
\end{equation*}
And
\begin{equation}
	\omega^- : V \times \mathcal{P}(G) \rightarrow \mathcal{P}(V) 
\end{equation}
Where
\begin{equation*}
	\omega^-(v, G) = 
	\{v \in V \mid \exists (u, v) \in E\}
\end{equation*}

\end{defn}

\medskip

\begin{defn} (Behavior functions)
\begin{align}
	& \beta_{s} : C \to (\mathcal{P}(S) \times S \times \mathbb{N}_{0} \to S) \\
	& \beta_{o} : C \to (S \times \mathbb{N}_{0} \to S)
\end{align}

\end{defn}

\medskip

\begin{defn} (Redstone Digraph)
	
	Let:
	
	\begin{itemize}
		\item \(G = (V, E)\) be any digraph
		\item \(\Sigma : V \times \mathbb{N}_{0} \to S \times \mathbb{N}_{0}\)
		\item \(\lambda : V \to C\)
	\end{itemize}

	Then \(G_{r}\) is a redstone digraph if 
	\begin{equation}
	G_{r} = (G, \lambda, \Sigma)
	\end{equation}
	
\end{defn}
	
\begin{defn} (State)

Let \(G_{r} = (G, \lambda, \Sigma)\) be any redstone digraph where \(G = (V, E)\).

For any vertex \(v \in V\) at an arbitrary tick \(t \in \mathbb{N}_{0}\), \(I_{v}\) is a set of Input Signals 
if

\begin{equation*}
	I_{v} = \{ ((\beta_{o} \circ \lambda)(v) \circ \Sigma)(u, t) \mid u \in \omega^-(v, G)  \}
\end{equation*}

and \(D_{v}\) is a set of Input Durations if

\begin{equation}
	D_{v} = \{\varphi \in S \mid (\varphi, \delta) \in \Sigma(v, t), \delta \in \mathbb{N}_{0}\} \cup \{\Phi\}
\end{equation}

Where

\begin{equation}
	\Phi = (\beta_{s} \circ \lambda)(v)(I_{v}, \Sigma(v, t))
\end{equation}

And

\begin{equation}
	\Delta = 
	\begin{cases}
		\delta + 1,\, (\varphi, \delta) \in \Sigma(v, t) & \text{if } (D_{v} \subseteq S^+) \vee (D_{v} \subseteq S_{0})\\	
		0 & \text{otherwise} \\	
	\end{cases}
\end{equation}

Then

\begin{equation}
	\Sigma(v, t + 1)|_{V \times \mathbb{Z}^+} = (\Phi, \Delta) \\
\end{equation}

Note: the state mapping isn't defined for \(t = 0\) to allow the implementation to define a custom mapping for all 
vertices at \(t = 0\).

\end{defn}

\section{Redstone Objects}

\begin{defn} (Redstone Torch)

	Let
	\begin{align}
		& T_{s} : \mathcal{P}(S) \times S \times \mathbb{N}_{0} \rightarrow S \\
		& T_{o} : S \times \mathbb{N}_{0} \rightarrow S
	\end{align}
	Where
	\begin{align}
		& T_{s}(I, \varphi, \delta) =
		\begin{cases}
			0 & \text{if } Max(I) \subseteq S_{0} \\
			15 & \text{otherwise}
		\end{cases} \\[\jot]
		& T_{o}(\varphi, \delta) =
		\begin{cases}
			15 & \text{if } \varphi \in S_{0} \\
			0 & \text{otherwise}
		\end{cases}
	\end{align}
	Then
	\begin{align}
		& \beta_{s}(T_{c}) = T_{s} \\
		& \beta_{o}(T_{c}) = T_{o}
	\end{align}
\end{defn}

\begin{defn} (Repeater(s))
	
	Let
	\begin{align}
		& n \in \{1, 2, 3, 4\}, Rn_{s} : \mathcal{P} (S) \times S \times \mathbb{N}_{0} \rightarrow S \\
		& n \in \{1, 2, 3, 4\}, Rn_{o} : S \times \mathbb{N}_{0} \rightarrow S
	\end{align}
	Where
	\begin{align}
		& Rn_{s}(I, \varphi, \delta) = 
		\begin{cases}
			15 & \text{if } Max(I) \subseteq S^+\\
			15 & \text{if } (\varphi \in S^+) \land (Max(I) \subseteq S_{0}) \land (\delta < n) \\
			0 & \text{ otherwise }
		\end{cases} \\[\jot]
		& Rn_{o}(\varphi, \delta) = 
		\begin{cases}
			15 & \text{if } (|\varphi \cap S^+| > 1) \land (\delta >= n) \\
			15 & \text{if } (\varphi \subseteq S_{0}) \land (\delta < n) \\
			0 & \text{ otherwise }
			\end{cases}
	\end{align}
	Then
	\begin{align}
			& n \in {1, 2, 3, 4}, \beta_{s}(Rn_{c}) = Rn_{s} \\
			& n \in {1, 2, 3, 4}, \beta_{o}(Rn_{c}) = Rn_{o}
	\end{align}
\end{defn}

\begin{defn} (Lamp (Output))
	
	Let
	\begin{align}
		& O_{s} : \mathcal{P} (S) \times S \times \mathbb{N}_{0} \rightarrow S \\
		& O_{o} : S \times \mathbb{N}_{0} \rightarrow S
	\end{align}
	Where
	\begin{align}
		&O_{s}(I, \varphi, \delta) = 
		\begin{cases}
			0 & \text{if } Max(I) \subseteq S_{0} \\
			15 & \text{otherwise}
		\end{cases} \\[\jot]
		&O_{o}(\varphi, \delta) = 
		\begin{cases}
			15 & \text{if } (\varphi \cap S^+| > 1) \\
			15 & \text{if } (\varphi \subseteq S_{0}) \land (\delta = 0) \\
			0 & \text{otherwise}
		\end{cases}
	\end{align}
	Then
	\begin{align}
		& \beta_{s}(O_{c}) = O_{s} \\
		& \beta_{o}(O_{c}) = O_{o}
	\end{align}
\end{defn}

\begin{defn} (Lever (Input))
	
	Let
	\begin{align}
		& I_{s} : \mathcal{P} (S) \times S \times \mathbb{N}_{0} \rightarrow S \\
		& I_{o} : S \times \mathbb{N}_{0} \rightarrow S
	\end{align}
	Where
	\begin{align}
		& I_{s}(I, \varphi, \delta) = \varphi \\
		& I_{o}(\varphi, \delta) = \varphi
	\end{align}
	Then
	\begin{align}
		& \beta_{s}(I_{c}) = I_{s} \\
		& \beta_{o}(I_{c}) = I_{o}
	\end{align}
\end{defn}

\begin{proof} Clock proof
	
	\begin{itemize}
		\item Let \(G_{r} = (G, \lambda, \Sigma)\) be a redstone digraph where \(G = (V, E)\). 
		\item Let \(V = \{t_{1}, t_{2}, t_{3}\}\) and \(E = \{(t_{1}, t_{2}), (t_{2}, t_{3}), (t_{3}, t_{1})\}\).
		\item Let \(\forall v \in V(\lambda(v) = T_{c})\).
		\item Let \(\forall v \in V(\Sigma(v, 0) = (0, D_{m}))\).
	\end{itemize}

	For \(t_{3}\) (vertex) at \(t = 1\) (tick), \(I_{t_{3}}\) is a set of Input Vertices where 
	\begin{align*}
		& I_{t_{3}} = \{((\beta_{o} \circ \lambda)(t_{3}) \circ \Sigma)(t_{2}, 0)\} \\
		& = \{(\beta_{o} \circ T_{c} \circ \Sigma)(t_{2}, 0)\} \\
		& = \{T_{o}(\Sigma(t_{2}, 0)\} \\
		& = \{T_{o}(0, 4)\} \\
		& = \{15\} \\
	\end{align*}

	Now let

	\begin{align*}
		& \Phi = (\beta_{s} \circ \lambda)(v_{3})(I_{v_{3}}, \Sigma(v_{3}, 0)) \\
		& = (\beta_{s} \circ T_{c})(\{15\}, 0, 4) \\
		& = T_{s}(\{15\}, 0, 4) \\
		& =  15 \\
	\end{align*}
	
	Then \(D_{t_{3}}\) is a set of Input Durations where
	\begin{align*}
		& D_{t_{3}} = \{\phi \in S \mid (\phi, \delta) \in \Sigma(v_{3}, 0)\} \cup \{\Phi\} \\
		& = \{0\} \cup \{15\} \\
		& = \{0, 15\} \\
	\end{align*}

	Now let 
	\begin{align*}
		\Delta = 0
	\end{align*}

	Then
	
	\begin{align*}
		& \Sigma(t_{3}, 1) = (\Phi, \Delta) \\
		& = (15, 0)
	\end{align*}

	And the output at \(t = 1\) is then 
	\begin{align*}
		(\beta_{o} \circ \lambda)(t_{3})(\Sigma(t_{3}, 1) &= (\beta_{o} \circ T_{c})(\Sigma(t_{3}, 1)) \\
		& = T_{o}(\Sigma(t_{3}, 1)) \\ 
		& = T_{o}(15, 0) \\ 
		& = 0 \\ 
	\end{align*}

	And the output at \(t = 0\) is 
	\begin{align*}
		(\beta_{o} \circ \lambda)(t_{3})(\Sigma(t_{3}, 0) &= (\beta_{o} \circ T_{c})(\Sigma(t_{3}, 0)) \\
		& = T_{o}(\Sigma(t_{3}, 0)) \\ 
		& = T_{o}(0, 4) \\ 
		& = 15 \\ 	
	\end{align*}

	Finally, it is then proven that

	\begin{displaymath}
		\begin{array}{|c|c|c|}
		t_{3} & t = 0 & t = 1\\
 		\hline
		0 & 15 & 0
 		\end{array}
	\end{displaymath}

	
\end{proof}

\end{document}

